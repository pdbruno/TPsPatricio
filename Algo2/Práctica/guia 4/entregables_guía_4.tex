
\documentclass[10pt, a4paper]{article}
\usepackage[spanish]{babel}
\usepackage[utf8]{inputenc}
\usepackage{aed2-symb,aed2-itef,aed2-tad,aed2-diseno, caratula,amssymb, parskip}
\usepackage{algorithmicx, algpseudocode, algorithm}

\begin{document}


\integrante{Bruno, Patricio Damián}{62/19}{pdbruno@gmail.com}
\materia{Algoritmos y Estructuras de Datos II}
\submateria{Guía 4}
\titulo{Ejercicios obligatorios de la práctica}



\maketitle

\section{Ordenamiento}

\begin{algorithm}[H]{sortSets(\In{A}{arreglo(ConjuntoDeNaturales))}) $\to$ $res$: arreglo(ConjuntoDeNaturales))}	
	\begin{algorithmic}[1]
			 \State $K \gets obtenerMaximoCardinal(A)$			\Comment $O(NK)$
			 \State $cardinales: arreglo\_dimensionable\; de\; lista(puntero(ConjuntoDeNaturales))$
			 \State $cardinales \gets crearArreglo(K + 1)$			\Comment $O(K)$
			 \For{$i\gets 1$ to $K+1$}			\Comment $O(K)$
			 	\State $cardinales[i]\gets Vacia()$			\Comment $O(1)$
			 \EndFor
			 \For{$i\gets 1$ to $tam(A)$}			\Comment $O(N)$
			 	\State $cardinal \gets obtenerCardinal(A[i])$			\Comment $O(K)$
			 	\State $AgregarAtras(cardinales[cardinal] + 1, \&A[i])$			\Comment $O(1)$
			 \EndFor
			 \State $res: arreglo\_dimensionable\; de\; ConjuntoDeNaturales$
			 \State $res \gets crearArreglo(tam(A))$	\Comment $O(N)$
			 \State $j \gets 1$                    \Comment $O(1)$
    	
    		\For{$i\gets 1$ to $K+1$}				\Comment $O(K)$
    			\State $it \gets CrearIt(cardinales[i])$
    			\While{$HaySiguiente?(it)$}			\Comment $O(N)$ en total
			 		\State $res[j]\gets *Siguiente(it)$	\Comment $O(copy(Siguiente(it)))$
			 		\State $j\gets j+1$	\Comment $O(1)$
			 		\State $Avanzar(it)$	\Comment $O(1)$
			 	\EndWhile
			\EndFor \\
			\Return $res$
		\medskip
		\Statex \underline{Justificación:} La idea es usar bucket sort, donde tengo K+1 buckets, siendo K el mayor cardinal de todos los conjuntos en A (el corrimiento se debe a que deben entrar cardinales entre 0 y K en un arreglo cuya indexación comienza en 1 ). En cada bucket tengo una lista de punteros a los conjuntos (para evitar el costo de copia) cuyo cardinal corresponde con el índice del bucket. Voy a \textbf{suponer} que el costo de copia de un conjunto, sabiendo que tiene a lo sumo K elementos, es O(K). A continuación voy a demostrar por qué el algoritmo a partir de la línea 14 tiene complejidad en peor caso de O(NK) y luego justificar la complejidad de todo el algoritmo. Notar que si un conjunto está contenido en otro, entonces su cardinal es menor, y con ese dato alcanza para ordenarlos según lo que pide la consigna. La última aclaración es que en este algoritmo y en los demás uso la sintaxis de declaración e inicialización de arreglos que aparece en el módulo de diseño de la cátedra
    \end{algorithmic}
\end{algorithm}	

Sea $A_i$ con $1 \leq i \leq N$ una sucesión o secuencia de conjuntos\\
Sabemos que $\forall i / 1 \leq i \leq N$, se tiene $\#A_i \leq K$\\
Sea $C_j$ con $0 \leq j \leq K$ una sucesión o secuencia de conjuntos tal que\\
$C_j = \{ \; C \; | \; \exists i \in \mathbb{N} / 1 \leq i \leq N \wedge A_i = C \wedge \#A_i = j \}$ \\
Por lo tanto $C_j$ constituye una partición de $A_i$ (hace las veces de buckets) y por lo tanto $\sum_{j=0}^{K}\#C_j = N$ \\ \\
Dicho esto, la complejidad a partir de la línea 14 se puede expresar de la siguiente manera: \\
$\sum_{j=0}^{K} O(1) + \#C_j * (O(K) + O(1)) = \sum_{j=0}^{K}\#C_j * O(K) =$\\
$O(K)\sum_{j=0}^{K}\#C_j = O(K)*N = O(NK)$
\\\\
 
$T_{peor}(N) = O(NK) + O(K) + O(K) + \sum_{i=1}^{N}(O(1) + O(K)) + O(N) + O(1) + O(NK)$\\
$= O(NK) + O(K) + N*O(K) + O(N) + O(1)$ \\
$= O(max(NK, K, N, 1))$\\
$= O(NK)$\\

\begin{algorithm}[H]{obtenerMaximoCardinal(\In{A}{arreglo(ConjuntoDeNaturales)}) $\to$ $res$: nat}	
	\begin{algorithmic}[1]
			 \State $res \gets -1$			\Comment $O(1)$
			 \For{$i\gets 1$ to $tam(A)$}			\Comment $O(N)$
			 	\State $card\gets obtenerCardinal(A[i])$			\Comment $O(K)$
			 	\State $res\gets max(res, card)$			\Comment $O(1)$
			 \EndFor \\
			\Return $res$
		\medskip
		\Statex \underline{Justificación:} Hago una operación con complejidad temporal en peor caso perteneciente a O(K) un total de N veces, por lo tanto la complejidad final es O(NK)
    \end{algorithmic}
\end{algorithm}	



\begin{algorithm}[H]{obtenerCardinal(\In{C}{ConjuntoDeNaturales}) $\to$ $res$: nat}	
	\begin{algorithmic}[1]
			\State $res \gets 0$			\Comment $O(1)$
    		\State $it \gets CrearIt(C)$	\Comment $O(1)$
    		\While{$HaySiguiente?(it)$}			\Comment $O(K)$
			 	\State $res\gets res+1$	\Comment $O(1)$
			 	\State $Avanzar(it)$	\Comment $O(1)$
			 \EndWhile \\
			\Return $res$
		\medskip
		\Statex \underline{Justificación:} Para cada elemento en C (a lo sumo K elementos) sumo 1 a res
    \end{algorithmic}
\end{algorithm}

\section{Dividir y Conquistar}
\subsection{}

\begin{algorithm}[H]{sonDisjuntos(\In{C}{arreglo(ConjuntoDeNaturales)}) $\to$ $res$: bool}	
	\begin{algorithmic}[1]
		\State $res \gets sonDisjuntosRec(C, 1, tam(C))$\\
		\Return $res.sonDisjuntos$
    \end{algorithmic}
\end{algorithm}	

\begin{algorithm}[H]{sonDisjuntosRec(\In{C}{arreglo(ConjuntoDeNaturales)}, \In{inicio}{nat}, \In{fin}{nat}) $\to$ $res$: $<sonDisuntos: bool, union: lista(nat)>$}	
	\begin{algorithmic}[1]
		\If{$fin - inicio = 1$}					\Comment $O(1)$
			\State $list \gets Vacia()$			\Comment $O(1)$
			 \For{$x$ in $C[inicio]$}			\Comment $O(M)$
			 	\State $AgregarAtras(list, x)$	\Comment $O(1)$
			 \EndFor \\
			\Return $<true, list>$
		\Else
			\State $med \gets (inicio + fin)/2$	\Comment $O(1)$
			\State $izq \gets sonDisjuntosRec(C, inicio, med)$	\Comment $O(T(n/2))$
			\State $der \gets sonDisjuntosRec(C, med, fin)$	\Comment $O(T(n/2))$
			\State $disjuntos \gets true$	\Comment $O(1)$
			\State $listaUnion \gets Vacia()$	\Comment $O(1)$
			\State $itIzq \gets CrearIt(izq.union)$	\Comment $O(1)$
			\State $itDer \gets CrearIt(der.union)$	\Comment $O(1)$
			
			\While{$(HaySiguiente?(itIzq) \vee HaySiguiente?(itDer)) \wedge disjuntos$}			\Comment $O(N*M)$
				\If{$HaySiguiente?(itIzq) \wedge HaySiguiente?(itDer) \wedge Siguiente(itIzq) = Siguiente(itDer)$}
				\State $disjuntos \gets false$	\Comment $O(1)$
				\EndIf \\
				\Comment el resto es el algoritmo clásico de merge de secuencias
				\If{$\neg HaySiguiente?(itDer) \vee (HaySiguiente?(itIzq) \wedge Siguiente(itIzq) < Siguiente(itDer))$}

					\State $AgregarAtras(listaUnion, Siguiente(itIzq))$	\Comment $O(1)$
					\State $Avanzar(itIzq)$	\Comment $O(1)$
				\Else
					\State $AgregarAtras(listaUnion, Siguiente(itDer))$	\Comment $O(1)$
					\State $Avanzar(itDer)$	\Comment $O(1)$
				\EndIf	 	
			\EndWhile \\
			
		\EndIf	\\
		\Return $<disjuntos \wedge izq.disjuntos \wedge der.disjuntos, listaUnion>$
	
		\medskip
		\Statex \underline{Justificación:} En este esquema de Divide \& Conquer, el caso base cuesta O(M), dividir cuesta O(1) y combinar los resultados cuesta O(NM). La idea es que de cada subproblema obtengo una lista ordenada de naturales que representa la unión de todos los conjuntos que entraban en ese subproblema. Entonces para los dos resultados de los subproblemas, aplico el algoritmo de mergeo de secuencias ordenadas, con la salvedad de que si encuentro dos elementos iguales, ya se que los conjuntos no son disjuntos. La cantidad de elementos en la unión de los N conjuntos es N*M, entonces en cada llamado recursivo el costo de combinar ambos subproblemas implica iterar sobre esos N*M elementos para generar la nueva unión. A continuación, el cálculo de complejidad en peor caso del algoritmo.
    \end{algorithmic}
\end{algorithm}	

$T(N)= \left\{ \begin{array}{lcc}
             2T(N/2) + O(NM) &   si  & N > 1 \\
             \\ O(M) &  si  & N = 1
             \end{array}
   \right.$\\ \\
$T(N) = 2T(\frac{N}{2}) + O(NM)\\ \\
 = 2(2T(\frac{N}{4}) + O(\frac{N}{2}M)) + O(NM) \\ \\
 = 2^2 T(\frac{N}{2^2}) + \sum_{j=0}^{2-1}O(\frac{N}{2^j} M)$\\ \\
 si seguimos expandiendo queda\\ \\
 $2^i T(\frac{N}{2^i}) + \sum_{j=0}^{i-1}O(\frac{N}{2^j} M)$\\ \\
 hasta que $i = log_{2}(N)$. También recordemos que T(1) es O(M) \\ \\
 $2^{log(N)} T(\frac{N}{2^{log(N)}}) + \sum_{j=0}^{log(N)-1}O(\frac{N}{2^j} M)\\ \\
 = N T(1) + \sum_{j=0}^{log(N)-1}O(\frac{N}{2^j} M) \\ \\
 = N * O(M) + \sum_{j=0}^{log(N)-1}O(\frac{N}{2^j} M) \\ \\
 = O(NM) + \sum_{j=0}^{log(N)-1}O(\frac{N}{2^j} M) \leq O(NM) + \sum_{j=0}^{log(N)-1}O(NM) \\ \\ 
 = O(NM) + (log(N) - 1)*O(NM) = (log(N) - 1 + 1) O(NM) = O(M N log(N))$


\subsection{}

\begin{algorithm}[H]{sonDisjuntos(\In{C}{arreglo(ConjuntoDeNaturales)}) $\to$ $res$: bool}	
	\begin{algorithmic}[1]
		\State $elementos: arreglo\_dimensionable\; de\; bool$
		\State $M \gets obtenerCardinal(C[1])$			\Comment $O(M)$		
		\State $elementos \gets crearArreglo(tam(C) * M))$	\Comment $O(N*M)$
		
		\For{$i\gets 1$ to $tam(elementos)$}			\Comment $O(N*M)$
			 \State $elementos[i] \gets false$			\Comment $O(1)$
		\EndFor
		
		\State $res \gets true$			\Comment $O(1)$
		
		\State $i \gets 1$	\Comment $O(1)$
		\While{$i \leq tam(C) \wedge res$}			\Comment $O(N)$
			\State $it \gets CrearIt(C[i])$	\Comment $O(1)$
			\While{$HaySiguiente?(it) \wedge res$}			\Comment $O(M)$
				\State $elem \gets Siguiente(it)$			\Comment $O(1)$	
				\If{$elementos[elem]$} 			\Comment $O(1)$	
					\State $res \gets false$	\Comment $O(1)$
				\Else
					\State $elementos[elem] \gets true$	\Comment $O(1)$
				\EndIf

				\State $Avanzar(it)$			\Comment $O(1)$		 	
			\EndWhile

			\State $i\gets i+1$			\Comment $O(1)$		 	
		\EndWhile \\
		
		\Return $res$
	
		\medskip
		\Statex \underline{Justificación:} Ahora aprovecho que los elementos de los conjuntos tienen una cota para crear una suerte de registro de los elementos que van apareciendo. El algoritmo crea un arreglo de N*M de largo (la cantidad de elementos distintos) y lo lleno con falses (todo esto es O(NM)). Luego, para todos los conjuntos y para todos los elementos en cada conjunto, si ya había encontrado ese elemento entonces sé que no son disjuntos, y en caso contrario seteo el registro en true para denotar que ya está presente en la unión de conjuntos
    \end{algorithmic}
\end{algorithm}	




\end{document}