\documentclass[10pt, a4paper]{article}
\usepackage[spanish]{babel}
\usepackage[utf8]{inputenc}
\usepackage{aed2-symb,aed2-itef,aed2-tad,aed2-diseno, amssymb, parskip}


\author{Algoritmos y Estructuras de Datos II, DC, UBA.}
\date{}
\title{Super TP AED2}

\begin{document}

\maketitle

\section{Complejidad}

\subsection{funciones polinomiales vs. funciones exponenciales}
\subsubsection*{(a) Sea $b \in \mathbb{R}$ tal que $b \ge 2$. Demostrar que $n \le b^n$ para todo $n \in \mathbb{N}$, por inducción en n.}
\paragraph*{Caso Base}
$n = 1, b \in \mathbb{R}$ \ / \ $b \ge 2$ \\
$1 \le b^1 = b \Leftrightarrow 1 \le 2 \le b$ \ al ser b mayor o igual a 2

\paragraph*{Hipótesis Inductiva}
$P(n) \Rightarrow P(n+1)$ \\
$n \le b^n \Rightarrow n+1 \le b^{n+1}$ \\
$n \le b^n \Leftrightarrow n+1 \le b^n+1$ \\
 Ahora quiero ver que \ $b^n+1 \le b^n*b$ \\
$b^n+1 \le b^n*b \underbrace{\Leftrightarrow}_{b \ne 0} 1+\frac{1}{b^n}\le b$ \\
$b \ge 2 \Leftrightarrow b^n \ge 2^n \Leftrightarrow \frac{1}{b^n} \le \frac{1}{2^n} \Leftrightarrow \boxed{1+\frac{1}{b^n} \le 1+\frac{1}{2^n}}$ \ Ahora quiero ver que \ $1+\frac{1}{2^n} \le b$ \\
$n \ge 1 \Leftrightarrow 2^n \ge 2 \Leftrightarrow \frac{1}{2^n} \le \frac{1}{2} \Leftrightarrow 1+\frac{1}{2^n} \le 1+\frac{1}{2} = \frac{3}{2} \le 2 \le b$ \\
Entonces $\boxed{1+\frac{1}{b^n} \le 1+\frac{1}{2^n} \le b}$ \\
Así queda demostrado que $n \le n+1 \le b^n+1 \le b^{n+1}$

\subsubsection*{(b) Sea $b \in \mathbb{R}$ tal que $b \ge 2$. Demostrar que $x\le b^{x+1}$ para todo $x\in \mathbb R_{\ge 0}$.}

$x\in \mathbb R_{\ge 0} \Rightarrow \underbrace{\lfloor x \rfloor \in \mathbb{N}}_{por definici\acute{o}n} \wedge  \ \lfloor x \rfloor \ge 0$ \\
$\lfloor x \rfloor \le x \le \lfloor x \rfloor + 1$ \ Ya que la parte entera de x se define como el mayor entero menor o igual a x. Esto implica que $\lfloor x \rfloor + 1$ es necesariamente entero (no negativo) y necesariamente mayor a x (si fuera menor o igual a x, estaríamos contradiciendo la definición de la parte entera)\\

Sea $n = \lfloor x \rfloor + 1$, por lo demostrado anteriormente sabemos que $n \le b^n$\\
Entonces $x \le \lfloor x \rfloor + 1 \le b^{\lfloor x \rfloor + 1} \le b^{x + 1}$

\subsubsection*{(c) Sea $b \in \mathbb{R}, k \in \mathbb{N}$ tal que $b > 1$ y $b^k \ge 2$. Demostrar que $\frac{x}{k}\le b^{x+k}$ para todo $x\in \mathbb R_{\ge 0}$.}

Por lo demostrado anteriormente, $\frac{x}{k} \le b'^{\frac{x}{k} + 1}$ para alguna base $b' \ge 2$, ya que $\frac{x}{k} \in \mathbb R_{\ge 0}$. Sea esa base $b^k$.\\
Entonces vale que $\frac{x}{k}\le (b^k)^{\frac{x}{k} + 1} = b^{x+k}$ para todo $x\in \mathbb R_{\ge 0}$

\subsubsection*{(c) Sea $b \in \mathbb{R}, k \in \mathbb{N}$ tal que $b > 1$ y $b^k \ge 2$. Demostrar que $(\frac{x}{pk})^n \le b^{n(\frac{x}{p} + k)}$ para todo $x\in \mathbb R_{\ge 0}$ y para todo $n, p \in \mathbb{N}$.}

\section{Invariante de representación y \\función de abstracción}



\end{document}